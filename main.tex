\documentclass[UTF8,zihao=-4,linespread=1.5]{ctexart}

\ctexset{
    abstractname = {\zihao{-2}\heiti 摘 \qquad 要} % 摘要标题格式
}

% Word 默认页边距
\usepackage[a4paper,top=1in,bottom=1in,left=1.25in,right=1.25in]{geometry}

% 页眉页脚
\usepackage{fancyhdr}
\pagestyle{plain}

% Times New Roman
\usepackage{times}

% 章节标题
\usepackage{titlesec}
\titleformat{\section}{\zihao{-2}\sffamily\heiti}{\thesection}{0.5em}{}
\titleformat{\subsection}{\zihao{-3}\sffamily\heiti}{\thesubsection}{0.5em}{}
\titleformat{\subsubsection}{\zihao{4}\sffamily\heiti}{\thesubsubsection}{0.5em}{}

% 图表标题
\usepackage{caption}
\DeclareCaptionFont{kaishu}{\zihao{5}\kaishu}
\captionsetup{font=kaishu}

% 其他
\usepackage{array}
\usepackage{graphicx}

\begin{document}

\begin{titlepage}

    \vspace*{1cm}

    \begin{figure}[h]
        \centering
        \includegraphics[width=\textwidth]{image/xm_pic1.png}
    \end{figure}

    \vspace*{\fill}

    \begin{center}
        \zihao{1} \kaishu \bfseries
        《这里填写课程名称》课程大作业/实验报告
    \end{center}

    \vspace{4cm}

    \begin{table}[h]
        \centering
        \zihao{3} \kaishu \bfseries
        \begin{tabular}{rp{16em}<{\centering}}
            题目: & 这里填写题目    \\
            \cline{2-2}
                & 题目太长就再加一行 \\
            \cline{2-2}
            院系: & 这里填写学院    \\
            \cline{2-2}
            专业: & 这里填写专业    \\
            \cline{2-2}
            班级: & 这里填写班级    \\
            \cline{2-2}
            姓名: & 这里填写姓名    \\
            \cline{2-2}
            学号: & 这里填写学号    \\
            \cline{2-2}
        \end{tabular}
    \end{table}

    \vspace{1cm}

    \begin{center}
        \zihao{3} \kaishu \bfseries
        20XX 年 X 月
    \end{center}

\end{titlepage}

\begin{abstract}

    摘要内容。

\end{abstract}

\newpage

% -------------------------------------------------------- %
\section{这是一级标题}
\label{sec:this_is_a_section}
% -------------------------------------------------------- %

这是一些文字。

% -------------------------------------------------------- %
\subsection{这是二级标题}
\label{subsec:this_is_a_subsection}
% -------------------------------------------------------- %

There are some words.

% -------------------------------------------------------- %
\subsubsection{这是三级标题}
\label{subsubsec:this_is_a_subsubsection}
% -------------------------------------------------------- %

一些 Mixed with some English 中文。

% -------------------------------------------------------- %
\section*{可有可无的参考文献}
\label{sec:references}
% -------------------------------------------------------- %

\end{document}
